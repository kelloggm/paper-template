\section{Technique}
\label{sec:technique}

\todo{This section or sections explains the new techniques we used to solve the problem.
  Often, you'll want to put multiple sections here: for example, you might have an overview
  followed by a section of proofs---but it will depend on the exact content of the paper.
  Optionally, it may be preceded by a ``Background'' section that gives relevant context:
  for example, in papers about new Checker Framework checkers, we often include a background
  section that explains what type qualifiers and pluggable types are, the
  annotation syntax, etc.}

\todo{Give all sections descriptive names, not something bland like
  ``Technique'' that could appear in any paper.  Make your section names
  work for you.}

\todo{A note about todos: when a co-author (especially a senior co-author) leaves a
  todo for you, especially in a technical or experimental section, there are two
  ways to address it:
  \begin{itemize}
  \item If the todo points out a valid issue with the text (e.g., if the todo says ``does Y include Z?'',
    the text suggests that the answer is ``no'', but the actual answer is ``yes''), then \emph{change the text and delete the todo}. Do not just add
    more todo text saying ``yes'': instead, address the issue in the text directly, so that the next
    reader doesn't have the same question.
  \item If the todo is incorrect (e.g., if the todo says ``does Y include Z?'', the text suggests ``no'', and the answer really is ``no''),
    then it \emph{may} be appropriate to edit the todo to answer the question (don't just delete the todo!).
    In this case, your senior co-author can confirm that the text is correct and then \emph{comment out the todo},
    so that if another senior co-author has the same concern in the future they will find the commented-out todo
    as they try to write their own. Sometimes, though, the best course of action is to edit the text to be more explicit:
    most questions that your senior co-authors ask are going to be similar to the concerns raised by reviewers.
  \end{itemize}
  }
