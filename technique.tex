\section{Technique}
\label{sec:technique}

\todo{This section or sections explains the new techniques we used to solve the problem.
  Often, you'll want to put multiple sections here: for example, you might have an overview
  followed by a section of proofs---but it will depend on the exact content of the paper.
  Optionally, it may be preceded by a ``Background'' section that gives relevant context:
  for example, in papers about new Checker Framework checkers, we often include a background
  section that explains what type qualifiers and pluggable types are, the
  annotation syntax, etc.}

\todo{Give all sections descriptive names, not something bland like
  ``Technique'' that could appear in any paper.  Make your section names
  work for you.}

\todo{Avoid ``anthromorphism'' when describing automated steps taken by tools. For example, avoid phrases like "we enforce immutability...", "we make eligible resource fields final whenever possible...", "we introduce a temporary variable...", etc., because \textbf{we} (the authors) do not actually do these things. Instead, our tool does them!

It's important to be precise about this in technical writing, because we want to distinguish between things that we (the authors) do manually and things that our tool does automatically. The use of "we" for things done automatically by the tool can incorrectly give the impression that we (the authors) had to be involved in those things directly, which is incorrect.
}
