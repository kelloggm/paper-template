%%% This template uses IEEE formatting, for IEEE-sponsored conferences (such
%%% as ICSE, every two years).  If you need ACM formatting, See `paper.tex`.

\documentclass[conference]{IEEEtran}

\usepackage{cite}
\usepackage{amsmath,amssymb,amsfonts}
\usepackage{algorithmic}
\usepackage{graphicx}
\usepackage{textcomp}
\usepackage{xcolor}

% tables
\usepackage{tabularx}
\usepackage{booktabs}

%\usepackage{algorithm}
%\usepackage[noend]{algpseudocode}

% xspace command
\usepackage{xspace}

% for the researchquestions environment
\usepackage{enumitem}

% From https://tex.stackexchange.com/questions/177025/
%% \makeatletter
%% \newcounter{algorithmicH}% New algorithmic-like hyperref counter
%% \let\oldalgorithmic\algorithmic
%% \renewcommand{\algorithmic}{%
%%   \stepcounter{algorithmicH}% Step counter
%%   \oldalgorithmic}% Do what was always done with algorithmic environment
%% \renewcommand{\theHALG@line}{ALG@line.\thealgorithmicH.\arabic{ALG@line}}
%% \makeatother

% lstlisting command
\usepackage{listings}
\usepackage[scaled]{beramono}
\newcommand*\LSTfont{\Small\fontencoding{T1}\ttfamily\SetTracking{encoding=*}{-60}\lsstyle}
\lstset{language=Java,
  frame=none,
  aboveskip=1.5pt,
  belowskip=0pt,
  showstringspaces=false,
  columns=flexible,
  basicstyle=\LSTfont,
  numbers=none,
  numberstyle=\tiny\color{black},
  keywordstyle=\color{black},
  commentstyle=\color{black},
  stringstyle=\color{black},
  breaklines=true,
  breakatwhitespace=true,
  tabsize=3,
  %emph={@NonNegative,@Positive,@GTENegativeOne,@LTLengthOf,@LTEqLengthOf,@IndexFor,@IndexOrHigh,@IndexOrLow,@HasSubsequence,@LessThan,@SameLen,@SearchIndexFor,@MinLen,@ArrayLen,@IntVal,@IntRange,@LengthOf,@UpperBoundUnknown,@LowerBoundUnknown,int,double,List,Map,Object,SerialDate,Long,Integer,DefaultPolarItemRenderer,LegendItem,PolarPlot,XYDataset,long,T,String,string,byte,InputStream,CategoryDataset,DatasetRenderingOrder,ArrayList,Entry,Values,Number,ValuesContract,ImmutableIntArray,Dataset,XYZDataset}, emphstyle=\color{blue}
}

% Graphics
% \usepackage{tikz}
% \usetikzlibrary{arrows,automata,positioning}

% Change font and line spacing for figure captions
\usepackage{setspace,caption}
\captionsetup{labelfont={small,bf}, textfont={small,bf,stretch=0.8}, labelsep=colon, margin=0pt}

\usepackage{flushend} % balanced columns on last page

\usepackage{url} % URLs; used in plume-bib

% cref command; best to load last
\usepackage{cleveref}
\newcommand{\crefrangeconjunction}{--}

%%
%% end of the preamble, start of the body of the document source.
\begin{document}

% Use a macro like this for your tool's name. Never actually use your tool's name
% in the text: always use the macro.
\newcommand{\tool}{the Type Inference Framework\xspace}
\newcommand{\Tool}{The Type Inference Framework\xspace}
\newcommand{\toolShort}{TIF\xspace}
%% Mike thinks it's better to use a more readable macro name, such as "\theIndexChecker" rather than "\tool".
%% Martin doesn't really care either way; you can create a macro for your tool's name or leave it as-is.

%%% Todo comments
\newcommand{\todo}[1]{{\color{red}\bfseries [[#1]]}}
%% Comment or uncomment this line.
%\renewcommand{\todo}[1]{\relax}

\newcommand{\manu}[1]{\todo{#1 --MS}}

% Don't show todo commands if this macro is defined.
\ifdefined\notodocomments
  \renewcommand{\todo}[1]{\relax}
\fi

% Use like: \ifanonymous{ANONYMOUS TEXT}\else{NON-ANONNYMOUS TEXT}\fi
% where the "\else{NON-ANONNYMOUS TEXT}" may be omitted.
\newif\ifanonymous
%% Comment or uncomment this line
\anonymoustrue

\newcommand{\anonurl}[1]{\ifanonymous URL removed for anonymity.\else\url{#1}\fi}
\newcommand{\footnoteanonurl}[1]{\footnote{\anonurl{#1}}}

% \|name| or \mathid{name} denotes identifiers and slots in formulas
\def\|#1|{\mathid{#1}}
\newcommand{\mathid}[1]{\ensuremath{\mathit{#1}}}
% \<name> or \codeid{name} denotes computer code identifiers
\def\<#1>{\codeid{#1}}
% \protected\def\codeid#1{\ifmmode{\mbox{\sf{#1}}}\else{\sf #1}\fi}
% \protected\def\codeid#1{\ifmmode{\mbox{\ttfamily{#1}}}\else{\ttfamily #1}\fi}
\protected\def\codeid#1{\ifmmode{\mbox{\smaller\ttfamily{#1}}}\else{\smaller\ttfamily #1}\fi}

% research question list, based on the answer to https://tex.stackexchange.com/questions/559305/how-to-format-for-two-column-research-question
\newlist{researchquestions}{enumerate}{1}
\setlist[researchquestions]{label*=\textbf{RQ\arabic*}}

\newcommand{\CalledMethodsBottom}{\<@Call\-ed\-Meth\-ods\-Bottom>\xspace}
\newcommand{\CalledMethods}{\<@Call\-ed\-Meth\-ods>\xspace}
\newcommand{\EnsuresCalledMethods}{\<@En\-sures\-Call\-ed\-Meth\-ods>\xspace}
\newcommand{\MustCall}{\codeid{@Must\-Call}\xspace}
\newcommand{\MustCallAlias}{\codeid{@Must\-Call\-Alias}\xspace}
\newcommand{\MustCallUnknown}{\codeid{@Must\-Call\-Unknown}\xspace}
\newcommand{\CreatesMustCallFor}{\<@Creates\-Must\-Call\-For>\xspace}
% Deprecated
\newcommand{\ResetMustCall}{\CreatesMustCallFor}

% "trule" stands for ``type rule''
\newcommand{\trule}[2]{\[\frac{#1}{#2}\]}
\newcommand{\truleinline}[2]{\ensuremath{#1\mathrel{\vdash}#2}}
\newcommand{\hastype}[1]{\mathbin{:}\trtext{#1}}
\newcommand{\trcode}[1]{\codeid{\smaller\smaller #1}}
\newcommand{\trtext}[1]{\mbox{\smaller\smaller #1}}
\newcommand{\trquoted}[1]{\trcode{"}#1\trcode{"}}


%%% Computed values

% For any number that's referenced in the text itself (and a table), create a macro like these rather than copy-pasting.
% These examples are from the WPI paper; you can delete them.

\newcommand{\numTypeSystems}{11\xspace} % Formatter, index, interning, lock, nullness, regex, resourceleak, signature, signedness, InitializedFields, Optional
\newcommand{\numModifiedTypeSystems}{2\xspace} % Formatter, Nullness (not Called Methods, because the postcondition code is general)
\newcommand{\numProjects}{12\xspace}
\newcommand{\numLOC}{88,680\xspace}
\newcommand{\numHumanAnnos}{803\xspace}
\newcommand{\percentInferred}{39\todo{check}\%\xspace}
\newcommand{\warningReductionPercent}{45\todo{check}\%\xspace}
\newcommand{\tsSpecificLoC}{61\todo{check}\xspace}

%%% Miscellaneous

\hyphenation{type-state}        % LaTeX defaults to "types-tate"

%%% Space-saving hacks

% Reduce indentation in lists.
\setlength{\leftmargini}{.75\leftmargini}
\setlength{\leftmarginii}{.75\leftmarginii}
\setlength{\leftmarginiii}{.75\leftmarginiii}

\newcommand{\prefigcaption}{\vspace{-5pt}}
\newcommand{\posttablecaption}{\vspace{-5pt}}

% Reduce the separation between figures and text.
\addtolength{\textfloatsep}{-.25\textfloatsep}
\addtolength{\dbltextfloatsep}{-.25\dbltextfloatsep}
\addtolength{\floatsep}{-.25\floatsep}
\addtolength{\dblfloatsep}{-.25\dblfloatsep}

\newcommand{\zph}{\phantom{0}}
\newcommand{\zzph}{\phantom{00}}

\newcommand{\ie}{i.e.,\xspace}
\newcommand{\eg}{e.g.,\xspace}


\title{Software Engineering Paper Template}

\author{\IEEEauthorblockN{1\textsuperscript{st} Given Name Surname}
\IEEEauthorblockA{\textit{dept. name of organization (of Aff.)} \\
\textit{name of organization (of Aff.)}\\
City, Country \\
email address or ORCID}
\and
\IEEEauthorblockN{2\textsuperscript{nd} Given Name Surname}
\IEEEauthorblockA{\textit{dept. name of organization (of Aff.)} \\
\textit{name of organization (of Aff.)}\\
City, Country \\
email address or ORCID}
\and
\IEEEauthorblockN{3\textsuperscript{rd} Given Name Surname}
\IEEEauthorblockA{\textit{dept. name of organization (of Aff.)} \\
\textit{name of organization (of Aff.)}\\
City, Country \\
email address or ORCID}
\and
\IEEEauthorblockN{4\textsuperscript{th} Given Name Surname}
\IEEEauthorblockA{\textit{dept. name of organization (of Aff.)} \\
\textit{name of organization (of Aff.)}\\
City, Country \\
email address or ORCID}
\and
\IEEEauthorblockN{5\textsuperscript{th} Given Name Surname}
\IEEEauthorblockA{\textit{dept. name of organization (of Aff.)} \\
\textit{name of organization (of Aff.)}\\
City, Country \\
email address or ORCID}
\and
\IEEEauthorblockN{6\textsuperscript{th} Given Name Surname}
\IEEEauthorblockA{\textit{dept. name of organization (of Aff.)} \\
\textit{name of organization (of Aff.)}\\
City, Country \\
email address or ORCID}
}

%% This command processes the author and affiliation and title
%% information and builds the first part of the formatted document.
\maketitle

%%
%% The abstract is a short summary of the work to be presented in the
%% article.
\begin{abstract}
\begin{abstract}
  \todo{This is an example paper outline.  It briefly describes the
    problem, the key ideas of the solution, and the experimental results.}

  \todo{For more advice about writing a paper, see
    \url{https://homes.cs.washington.edu/~mernst/advice/write-technical-paper.html}.}
\end{abstract}


\label{dummy-label-for-etags:abstract}

\end{abstract}

%%
%% Keywords. The author(s) should pick words that accurately describe
%% the work being presented. Separate the keywords with commas.
\begin{IEEEkeywords}
datasets, neural networks, gaze detection, text tagging
\end{IEEEkeywords}

\section{Introduction}
\label{sec:intro}

\todo{
This paper outline gives the general structure of a conference paper about
a new practical verification tool, in the style that is standard for a
software engineering conference such as ICSE or FSE\@. Your paper
may have more sections (or fewer) depending on its topic, but any
SE paper will always have:
\begin{itemize}
\item an introduction, which lays out the problem to be solved, the new technique
used to solve the problem, and brags about how successful we were
(\cref{sec:intro});\footnote{The introduction should always end with a list
of contributions structured like this one.}
\item a description of the technique (\cref{sec:technique});
\item an evaluation section, whose results justify the claims in the
introduction (\cref{sec:evaluation});
\item a limitations or threats to validity section that explains in what
ways the experiments might be misleading, and what we've done to mitigate
those threats (\cref{sec:limitations}); and
\item a related work section that places the work in context (\cref{sec:relatedwork}).
\item a conclusion that recaps the contributions of the paper.
\item often, a short ``data availability'' that indicates whether and where
  your data and experimental scripts are available.
\end{itemize}
}



\section{Technique}
\label{sec:technique}

\todo{This section or sections explains the new techniques we used to solve the problem.
  Often, you'll want to put multiple sections here: for example, you might have an overview
  followed by a section of proofs---but it will depend on the exact content of the paper.
  Optionally, it may be preceded by a ``Background'' section that gives relevant context:
  for example, in papers about new Checker Framework checkers, we often include a background
  section that explains what type qualifiers and pluggable types are, the
  annotation syntax, etc.}

\todo{Give all sections descriptive names, not something bland like
  ``Technique'' that could appear in any paper.  Make your section names
  work for you.}

\todo{Avoid ``anthromorphism'' when describing automated steps taken by tools. For example, avoid phrases like ``we enforce immutability...'', ``we make eligible resource fields final whenever possible...'', ``we introduce a temporary variable...'', etc., because \textbf{we} (the authors) do not actually do these things. Instead, our tool does them!

It's important to be precise about this in technical writing, because we want to distinguish between things that we (the authors) do manually and things that our tool does automatically. The use of ``we'' for things done automatically by the tool can incorrectly give the impression that we (the authors) had to be involved in those things directly, which is incorrect.
}

\todo{A note about todos: when a co-author (especially a senior co-author) leaves a
  todo for you, especially in a technical or experimental section, there are two
  ways to address it:
  \begin{itemize}
  \item If the todo points out a valid issue with the text (e.g., if the todo says ``does Y include Z?'',
    the text suggests that the answer is ``no'', but the actual answer is ``yes''), then \emph{change the text and delete the todo}. Do not just add
    more todo text saying ``yes'': instead, address the issue in the text directly, so that the next
    reader doesn't have the same question.
  \item If the todo is incorrect (e.g., if the todo says ``does Y include Z?'', the text suggests ``no'', and the answer really is ``no''),
    then it \emph{may} be appropriate to edit the todo to answer the question (don't just delete the todo!).
    In this case, your senior co-author can confirm that the text is correct and then \emph{comment out the todo},
    so that if another senior co-author has the same concern in the future they will find the commented-out todo
    as they try to write their own. Sometimes, though, the best course of action is to edit the text to be more explicit:
    most questions that your senior co-authors ask are going to be similar to the concerns raised by reviewers.
  \end{itemize}
  }


\section{Implementation}
\label{sec:impl}

\todo{Any experimental paper about a new tool should include an ``implementation'' section.
  It may discuss a new tool you have built and/or the experimental infrastructure.
  Discuss the key non-obvious design decisions and complications.
  You should also mention: 1) what language the tool targets, 2) any important libraries that
  the tool relies on (\eg the Checker Framework), and 3) that the tool is
  open-source.

  Sometimes this section is very short.}


\section{Evaluation}
\label{sec:evaluation}

\todo{All experimental papers need a good evaluation section.
  This section might include a list of research questions.
  Often, there are separate sections for the experimental methodology and
  the results.

  One of the first things that I do when
  drafting the outline for a new paper is to design the ``main table'': that is, I add
  to this section a table that has row and column headers (but not actual numbers). I find this
  forces me to think about what I'm actually planning to measure, which helps design both
  better narratives (for the intro) and better experiments. This section might also be split
  up into multiple sections if there is a logical grouping of the experiments (\eg open- vs.\
  closed-source subjects, comparisions with other tools, etc.).}

\todo{It's common to include ``research questions'' in the structure
  of your evaluation narrative.  You can use the \<\textbackslash
  researchquestions> environment to automatically number and format
  them, like this:
  \begin{researchquestions}
  \item How do you write a good evaluation section?
  \end{researchquestions}
}


\section{Limitations and Threats to Validity}
\label{sec:limitations}

\todo{This section is required in SE papers. You should familiarize yourself
  with the terms ``internal'', ``external'', and ``construct'' threats to
  validity, and be prepared to use them to discuss your experiments in this
  section.

  One description appears at
  \url{https://github.com/mernst/uwisdom/blob/wiki/Research.adoc} (search
  for ``Threats'') but perhaps it can be improved.}


\section{Related Work}
\label{sec:relatedwork}

\todo{A related work section is not just a list of similar papers:
  its goal is to place this paper in context amongst the larger field.
  That means that anything you discuss in this section needs to not
  only be \emph{explained} on its own, but also must be \emph{compared}
  to the work described in this paper. When drafting, I usually start
  with just a list of related papers, and create the real related work
  section towards the end of the process (once the work in this paper has
  stabilized).}

\todo{
Write a citation like ``WPI\textasciitilde\textbackslash
cite\{KelloggDNAE2023\}''
(which formats as ``WPI~\cite{KelloggDNAE2023}'')
and not like ``WPI\textbackslash cite\{KelloggmDNAE2023\}''
(which formats as ``WPI\cite{KelloggDNAE2023}'').
That is, precede each use of \<\textbackslash cite> by a non-breaking
space, which is written as ``\<\textasciitilde>'' in a \<.tex> file.
}


\section{Conclusion}

\todo{Recap the contributions of the paper.  (In fact, it is good to think of
  this as a ``contributions'' section even if it's titled ``conclusion''.)
  This section can do so in more depth or with more reference to techniques
  and insights than the abstract and introduction were able to do, since
  readers of those sections wouldn't have read the whole paper yet.
  In other cases, this section is brief.}


\todo{The page limit is {\conferencePageLimit} pages.  This is page \thepage.}

%% The next two lines define the bibliography style to be used, and
%% the bibliography file.
\bibliographystyle{IEEEtran}
\bibliography{local,plume-bib/bibstring-abbrev,plume-bib/types,plume-bib/dispatch,plume-bib/ernst,plume-bib/soft-eng,plume-bib/crossrefs}

%%
%% If your work has an appendix, this is the place to put it.

\end{document}
\endinput
%%
